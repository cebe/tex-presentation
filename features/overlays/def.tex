%
% overlays
%

% http://tex.stackexchange.com/questions/16357/how-can-i-position-an-image-in-an-arbitrary-position-in-beamer
\usepackage{tikz}
%\usetikzlibrary{calc, trees, positioning, arrows, shapes, shapes.multipart, shadows, matrix, decorations.pathreplacing, decorations.pathmorphing}
\usetikzlibrary{arrows, shapes, shapes.multipart, automata}
\tikzset{
	every boverlay node/.style={
		draw=black,fill=white,rounded corners,anchor=north west,
	},
}
\tikzset{
	every toverlay node/.style={
		anchor=center
	},
}
\tikzset{
	every overlay node/.style={
		draw=white,fill=white,rounded corners,anchor=north west,
	},
}
% Usage:
% \tikzoverlay at (-1cm,-5cm) {content};%
% or
% \tikzoverlay[text width=5cm] at (-1cm,-5cm) {content};%
\def\tikzboverlay{% A bordered overlay
	\tikz[baseline,overlay]\node[every boverlay node]
}%
\def\tikztoverlay{% A transparent overlay
	\tikz[baseline,overlay]\node[every toverlay node]
}%
\def\tikzoverlay{% A non-transparent unbordered overlay
	\tikz[baseline,overlay]\node[every overlay node]
}%

